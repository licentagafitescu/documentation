\chapter*{Introducere} 
\addcontentsline{toc}{chapter}{Introducere}

În zilele noastre, datorită faptului că  majoritatea telefoanelor mobile au și funcția de cameră foto, imaginile au devenit principala formă de rememorare a evenimentelor trăite. Platforme întregi sunt dedicate fotografiilor, printre cele mai populare numărându-se Instagram, Flickr sau Pinterest. Multe din aceste platforme oferă posibilitatea de sortare, respectiv căutare a imaginilor în funcție de locație, mărime, data fotografiei sau descrierea adăugată de utilizator. Uneori aceste opțiuni nu sunt de ajuns, imaginile având mai multă relevanță din prisma conținutului decât după orice alte metadate. Totuși, opțiunea de a căuta imagini după similaritatea vizuală față de altă imagine, cu alte cuvinte, chiar după componența imaginii este o caracteristică rar întâlnită.

Aplicația {\applicationtitle} își propune implementarea acestei caracteristici \linebreak pentru platforma \textit{Flickr}, oferind utilizatorului posibilitatea de a vedea imagini similare vizual cu o imagine încărcată atât de pe propriul profil, cât și de pe profilele contactelor sau chiar din imaginile publice de pe platformă.

Scopul {\applicationtitle} este de a fi o unealtă utilă atât fotografilor care folosesc platforma (pentru a vedea lucrările din propriul profil care au aceeași tematică), cât și utilizatorului uzual (pentru a găsi fotografii legate de un anumit eveniment din viața sa, sau fotografii cu unele tematici alese).

Pentru realizarea funcționalității descrise au fost folosite diverse tehnologii web, precum Flak, Angular, împreună cu API-ul platformei \textit{Flickr} și un model de rețea neurală specializată în clasificarea imaginilor, model disponibil prin \textit{framework}-ul Tensorflow.

Capitolul \textit{Fundamente} va oferi o descriere a principalelor concepte, tehnologii și biblioteci folosite în implementarea aplicației. 

Scopul urmărit, o privire de ansamblu asupra aplicației și comparația cu unele soluții deja existente se vor regăsi în capitolul \textit{Scopuri și cerințe}.

Următorul capitol, \textit{Analiză și proiectare} conține descrierea arhitecturii aplicației, prezentarea componentelor necesare și a modului în care interacționează, precum și scenariile de utilizare ale aplicației.

În capitolul \textit{Detalii de implementare} se prezintă aspecte privitoare la modul de implementare, motivarea alegerii modelului de rețea neurală,  algoritmii și structurile de date folosite pentru a optimiza timpul de răspuns al aplicației precum și API-urile externe folosite.

Capitolul \textit{Manual de utilizare} prezintă modul de utilizare a aplicației
, fiind urmat de \textit{Concluzii}, o serie de concluzii legate de modul în care aplicația răspunde cerințelor și posibile direcții de dezvoltare ulterioară.
