\chapter{Analiză și proiectare}

\section{Scenarii de utilizare}

Utilizatorul poate efectua următoarele acțiuni:
\subsection{Autentificare}
Utilizatorul se autentifică în aplicație, care îl va redirecta către platforma Flickr, pentru a efectua autentificarea prin intermediul acesteia.

\subsection{Alege sursa imaginilor}
Utilizatorul poate alege sursa imaginilor ce vor fi analizare: propriul profil, imaginile contactelor sale, sau imagini publice de pe platforma Flickr.

\subsection{Încarcă imagine}
Pentru a efectua această acțiune, utilizatorul trebuie să fi trecut prin pasul de autentificare și eventual să fi ales sursa imaginilor. Dacă acesta nu a ales sursa imaginilor, în mod implicit aceasta este propriul profil. Scenariul acesta de utilizare presupune ca utilizatorul să aleagă o imagine de pe propriul dispozitiv pe care să o trimită către aplicație.

Pentru a returna imaginile similare din punct de vedere vizual, aplicația va efectua următoarele acțiuni:
\begin{enumerate}
    \item Analizează imaginea încărcată folosind o rețea neurală și returnează o serie de etichete descriptive
    \item Aduce imaginile  din sursa selectată de utilizator
    \item Analizează imaginile aduse, obținând pentru fiecare o serie de etichete
    \item Descoperă similaritatea între imagini comparând etichetele și obține o listă de imagini corespunzătoare
    \item Afișează imaginile din lista obținută
\end{enumerate}{}

 \begin{figure}[!htbp]
    \begin{center}
        \includegraphics[width=0.9\textwidth]{images/use_case.png}
        \caption{Diagrama de scenarii de utilizare a aplicației}
    \end{center}
\end{figure}


\section{Arhitectura aplicației}
Aplicația este împărțită în două module, dar folosește și anumite servicii externe. Utilizatorul interacționează direct cu modulul  \textit{front-end}, o aplicație web realizată cu ajutorul \textit{framework}-ului Angular în limbajul Typescript. La rândul său, modulul este împărțit în mai multe componente, acestea apelând unele servicii. Modulul de back-end este realizat cu ajutorul \textit{framework}-ului Flask în limbajul Python. Se folosesc mai multe biblioteci pentru a efectua anumite operații(ex. \textit{requests} pentru a apela API-uri, \textit{pony.orm} pentru conectarea la baza de date) și \textit{framework}-uri specializate pentru rețele neuronale(Tensorflow și Keras). Serviciile externe sunt reprezentate de API-urile platformelor Flickr(pentru autentificare, informații despre profil și imagini) și DataMuse(pentru a obține sinonime). De asemenea se mai folosește o bază de date SQLite pentru reținerea \textit{token}-ului de acces și identificatorului unic.

 \begin{figure}[!htbp]
    \begin{center}
        \includegraphics[width=1.0\textwidth]{images/arhitectura.png}
        \caption{Arhitectura aplicației}
    \end{center}
\end{figure}


Modulul \textit{front-end} este împărțit în 3 componente: HomeComponent, LoginComponent și ImageUploadComponent, iar în cadrul modulului se folosesc și 3 servicii: AuthService, Auth-GuardService și ImageService. HomeComponent are rolul de a oferi informații despre aplicație și a facilita procesul de autentificare. LoginComponent se ocupă de partea de autentificare în aplicație, folosind AuthService pentru a stoca anumite informații despre utilizator și a pregăti o experiență personalizată pentru acesta. ImageUploadComponent facilitează interacțiunea cu utilizatorul în cadrul scenariului de utilizare de încărcare a unei imagini. Folosind ImageUploadService, procesează cererile utilizatorului și răspunsul modulului de \textit{back-end}, la final afișând informațiile cerute de către utilizator. AuthGuardService este folosit în procesul de securizare a aplicației, împiedicând accesarea ImageUploadComponent de către un utilizator neautentificat.

 \begin{figure}[!htbp]
    \begin{center}
        \includegraphics[width=1.0\textwidth]{images/front_end.png}
        \caption{Arhitectura modulului front-end}
    \end{center}
\end{figure}

\pagebreak
Modulul \textit{back-end} este o aplicație web \cite{python-web-applications} ce expune un API , împărțită în 3 componente principare: main.py, prediction.py și repository.py. Submodulul main are mai multe sarcini: 
\begin{enumerate}
    \item Autentificarea cu Oauth
    \item Apelarea FlickrAPI
    \item Apelarea DataMuse API
    \item Comunicarea cu modulul \textit{front-end}
    \item Compararea listelor de etichete și detectarea similarității
\end{enumerate}{}
Submodulul repository este folosit pentru comunicarea cu baza de date, folosindu-se biblioteca de cod pony.orm în acest scop.
Submodulul prediction se ocupă de analiza imaginilor folosind o rețea neuronală și returnează lista de etichete ce descriu fiecare imagine în parte.

 \begin{figure}[!htbp]
    \begin{center}
        \includegraphics[width=1.0\textwidth]{images/back_end.png}
        \caption{Arhitectura modulului back-end}
    \end{center}
\end{figure}


\pagebreak
\section{Comunicarea între componentele aplicației}
Utilizatorul va comunica cu modulul de front-end pentru a trimite cereri și a primi informațiile corespunzătoare cererii sale. De asemenea utilizatorul va interacționa cu pagini afișate de Flickr în browser pentru a autoriza aplicația {\applicationtitle}.
Modulul front-end comunică cu utilizatorul și cu modulul back-end. Comunicare cu modulul back-end se realizează prin trimiterea de cereri HTTP \cite{http-requests} către un API pus la dispoziție de către back-end.
Pe lângă comunicarea cu modulul front-end deja menționată, modulul back-end va mai apela doua API-uri: API-ul Flickr pentru preluarea de imagini și informații despre utilizator precum și autentificarea prin protocolul OAuth 1.0 și DateMuse API pentru preluarea de cuvinte similare ca înțeles unui cuvânt dat.


\begin{figure}[!htbp]
    \begin{center}
        \includegraphics[width=1.0\textwidth]{images/secventa.png}
        \caption{Diagrama de secvență a aplicației}
    \end{center}
\end{figure}

\pagebreak
\section{Modulul \textit{back-end}}

\subsection{API-ul expus}
Un API (Application Programming Interface) este un set de rutine, protocoale  și unelte ce specifică modul în care componentele software interacționează.\cite{what-is-api}

Specificația OPENApi\cite{open-api} a API-ului se găsește la ~\nameref{ch:anexa1}
\subsubsection{/login}
Endpoint pentru autentificare, utilizatorul este redirectat aici de către modulul \textit{front-end}. Aplicația va redirecta la rândul ei catre /authorize.

\subsubsection{/authorize}
Endpoint pentru autorizare, se ajunge aici din redirectarea efectuată de către aplicație. Are ca parametri \textit{oauth-token} și \textit{oauth-verifier} și va redirecționa către \textit{front-end}.

\subsubsection{/profile}
Returnează informațiile despre profilul de Flickr al utilizatorului: numele utilizatorului și imaginea sa de profil. Are ca parametru identificatorul utilizatorului.

\subsubsection{/addImage}
Primește identificatorul utilizatorului, imaginea în base64 și opțiunea privitoare la sursa imaginilor ce vor fi analizate. Returnează lista de etichete a imaginii și lista de imagini similare din punct de vedere vizual.

\subsection{Rețeaua neurală folosită}
Rețeaua neurală folosită pentru analiza imaginilor este un model numit \textit{NASNet Mobile}, obținut de Google prin \textit{framework}-ul NAS(Neural Architecture Search) și discutat în lucrarea \textit{Learning Transferable Architectures for Scalable Image Recognition}. \cite{DBLP:journals/corr/ZophVSL17}

Proiectul \textit{AutoML}, prin \textit{framework}-ul NAS, propune următorul lucru: o rețea neurală controler propune o arhitectură de rețea "copil" care va fi antrenată și evaluată pe o anumită sarcină. Evaluarea este folosită pentru a calibra rețeaua controler, care va propune idei îmbunătățite de arhitectură la iterația următoare. Procesul se repetă de mii de ori, generând noi arhitecturi , testându-le și evaluându-le până când se obțin arhitecturi de rețele neurale noi utile.\cite{2016arXiv161101578Z}

\begin{figure}[!htbp]
    \begin{center}
        \includegraphics[width=0.6\textwidth]{images/nas.png}
        \caption{Procesul de obținere a noi rețele neurale folosind NAS \cite{2016arXiv161101578Z}}
    \end{center}
\end{figure}

Arhitectura \textit{NASNet} este obținută prin procesul descris mai sus, cu unele mic modificări. Scopul rețelei e să clasifice imagini, deci a fost evaluată performanța pe doua din cele mai cunoscute seturi de date în mediul academic: setul de detecție de obiecte \textit{COCO} și setul de clasificare de imagini \textit{ImageNet}. Deoarece seturile acestea de date sunt foarte mari, AutoML a fost reprogramat să caute straturi care ar fi atunci când sunt conectate în număr mare. Căutarea straturilor a avut loc pe un set de date mai mic, CIFAR10, apoi cea mai bună arhitectură a fost scalată și transferată pentru antrenamentul pe COCO și ImageNet.\cite{DBLP:journals/corr/ZophVSL17}



\begin{figure}[!htbp]
    \begin{center}
        \includegraphics[width=0.75\textwidth]{images/nasnet.png}
        \caption{Arhitectura NASNet \cite{DBLP:journals/corr/ZophVSL17}}
    \end{center}
\end{figure}
Arhitectura \textit{NASNet} a fost predefinită, având două tipuri de straturi:
\begin{enumerate}
    \item \textit{Normal Cell} - un strat convoluțional care returnează o mapare de aceași dimensiune a imaginii
    \item \textit{Reduction Cell} - un strat convoluțional care reduce dimensiunea mapării cu un factor de doi
\end{enumerate}{}
Rețeaua neurală controler a trebuit doar să descopere cea mai bună arhitectură pentru cele două tipuri de straturi.

După aproximativ 4 zile de căutări au fost obținute mai multe arhitecturi de straturi candidat, care combinate produc arhitecturi cu rezultate comparabile cu alte arhitecturi din domeniu. Cele mai bune 3 arhitecturi obținute se numesc \textit{NASNet-A}, \textit{NASNet-B} și \textit{NASNet-C}.

\begin{figure}[!htbp]
    \begin{center}
        \includegraphics[width=1.0\textwidth]{images/nasnetcells.png}
        \caption{Celulele arhitecturii NASNet-A \cite{DBLP:journals/corr/ZophVSL17}}
    \end{center}
\end{figure}

Modelul folosit în aplicația {\applicationtitle}, numit \textit{NASNet Mobile}, este o versiune scalată a arhitecturii \textit{NASNet}, mai mică, cu 5 326 716 parametri optimizată pentru un const computațional redus și o viteză mai mare. Modelul folosit este pre-antrenat pe 1000 de categorii din setul de date ImageNet, având o acuratețe de 91.9 procente pe același set de date.

\subsection{Baza de date}
Baza de date folosită, de tip SQLite, are un singur tabel folosit pentru a stoca \textit{token}-ul de acces și identificatorul unic al utilizatorului generat în relație cu acesta.

\begin{figure}[!htbp]
    \begin{center}
        \includegraphics[width=0.4\textwidth]{images/database.png}
        \caption{Schema bazei de date folosite}
    \end{center}
\end{figure}